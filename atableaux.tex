\documentclass{article}
\usepackage[svgnames]{xcolor}
\usepackage{atableaux}

\usepackage{listings}
\lstdefinestyle{latexcode}{
  language=[LaTeX]TeX,
  texcsstyle=*\bfseries\color{Blue},
  texcl=true,
  backgroundcolor=\color{Ivory},
  numbers=none,
  breaklines=true,
  framerule=2pt,
  keywordstyle=\color{Blue},
  commentstyle=\color{Brown},
  tabsize=2,
  morekeywords={Diagram, Tableau, EAbacus, braid},
  resetmargins=true,
}
\lstnewenvironment{latexcode}{\lstset{style=latexcode}}{}
\NewDocumentCommand\LaTeXCode{v}{\lstinline[style=latexcode]|#1|}

\newcommand\heading[1]{\bigskip\noindent\textsf{\large #1}}


%%%%%%%%%%%%%%%%%%%%%%%%%%%%%%%%%%%%%%%%%%%%%%%%%%%%%%
% braid diagrams:
%
% This isn't great but does what I mentioned
%
% Usage:
%  - \braid{a_1,...a_n}   : braid diagram connecting k to a_k
%  - \braid*{a_1,...a_n}  : braid diagram connecting k to a_k with top labels
%  - \braid!{a_1,...a_n}  : braid diagram connecting k to a_k with bottom labels
%  - \braid*!{a_1,...a_n} : braid diagram connecting k to a_k with top and bottom labels
% The command also accepts an optional argument, which is passed to the
% underlying TikZ environment.

\NewDocumentCommand\braid{ O{} s t! m }{%
  \begin{tikzpicture}[rounded corners, thick, NavyBlue, #1]
    \foreach \pt [count=\c] in {#4}{
      \node[atableaux/bead] at (\c,1){};
      \node[atableaux/bead] at (\pt,0){};
      \draw[atableaux/wall] (\c,1) -- (\pt,0);
    }
    \IfBooleanT{#2}{ \foreach \pt [count=\c] in  {#4} \node at (\c, 1.3) {$\c$}; }
    \IfBooleanT{#3}{ \foreach \pt [count=\c] in  {#4} \node at (\c,-0.3) {$\c$}; }
  \end{tikzpicture}%
}



\begin{document}

  \heading{Abacuses}

  Use the \LaTeXCode{\EAbacus} command to a typeset an abacus

  \begin{latexcode}
     Usage:  \textbackslash EAbacus{e}{rows}{partition}
     \EAbacus{3}{4}{5,4,1,1}   % \EAbacus{3}{4}{5,4,1,1}
  \end{latexcode}

  \heading{Braid diagrams}

  Use the \LaTeXCode{\braid} command to typeset briad diagrams:

  \begin{latexcode}
       \braid{1, 4, 2, 3}     % \braid{1, 4, 2, 3}
       \braid*{3, 1, 4, 2}    % \braid*{3, 1, 4, 2}
       \braid!{2, 3, 4, 1}    % \braid!{2, 3, 4, 1}
       \braid*!{4, 3, 2, 1}   % \braid*!{4, 3, 2, 1}
  \end{latexcode}

  \[
         \braid[scale=2]*!{4,3,2,1} =
  \]

                \braid*{3,2,4,1}\braid!{2,3,1,4}

  \heading{Taboids}

  Use the \LaTeXCode{\Tableau} command to typeset a tabloids:

  \begin{latexcode}
    \Tabloid{{1,3,7,9,11},{2,4,9,10},{6},{8}}   % \Tabloid{{1,3,7,9,11},{2,4,9,10},{6},{8}}
  \end{latexcode}

  \heading{Tableaux}

  Use the \LaTeXCode{\Tableau} command to a typeset tableau:

  \begin{latexcode}
    \Tableau{{1,3,7,9,11},{2,4,9,10},{6},{8}}   % \Tableau{{1,3,7,9,11},{2,4,9,10},{6},{8}}
  \end{latexcode}

  \begin{latexcode}
      \Tableau{ {,\bullet,,,,},{\bullet,B,\bullet,},{,\bullet,,},{,},{}}
      %\Tableau{ {,\bullet,,,,},{\bullet,B,\bullet,},{,\bullet,,},{,},{}}
  \end{latexcode}

  \heading{Young diagrams}

  Use the \LaTeXCode{\Diagram} command to typeset Young diagrams:

  \begin{latexcode}
       \Diagram{5,4,1,1}     % \Diagram{5,4,1,1} %
  \end{latexcode}


\end{document}
